%%
%% Copyright (c) 2018 Weitian LI <wt@liwt.net>
%% CC BY 4.0 License
%%
%% Resume / Résumé
%% A short document (1-2 pages) to sum up the job-related accomplishments
%% and experience.
%%
%% Information Checklist:
%% * Contact Information
%% * Work History / Experience
%% * Education
%% * Skills
%% * Summary & Objective (optional)
%% * Hobbies & Interests (optional)
%%
%% References:
%% * CV vs. Resume: What is the Difference? When to Use Which?
%%   https://uptowork.com/blog/cv-vs-resume-difference
%% * How to Make a Resume: A Step-by-Step Guide (+30 Examples)
%%   https://uptowork.com/blog/how-to-make-a-resume
%% * Entry-Level Resume: Sample and Complete Guide (+20 Examples)
%%   https://uptowork.com/blog/entry-level-resume-example
%%
%% 2018-04-11
%%

% Chinese version
\documentclass[zh]{resume}

% File information shown at the footer of the last page
\fileinfo{%
  \faCopyright{} 2018 ZhongJun Zhang,
  \creativecommons{by}{4.0},
  \githublink{zhangzhongjun}{resume}
}

\name{中俊}{张}
%\setposition{云计算开发工程师}
%\setlocation{深圳}

%\tagline{\getposition{} @ \getlocation}
\taglineicon{\faBinoculars}
%\keywords{\getposition, \getlocation, Python, Linux, BSD}
\photo{7em}{photo}

\socialinfo{
  \mobile{188-3510-9707}
  \email{18835109707@163.com}
  \github{zhangzhongjun} \\
  \university{西安电子科技大学}
  \degree{网络空间安全 \textbullet 硕士} \\
  %\address{上海}
  %\home{湖南 \textbullet 邵阳}
  %\birthday{1991-09-26}
}

\begin{document}
\makeheader

%======================================================================
% Summary & Objectives
%======================================================================
%{\onehalfspacing\hspace{2em}%
%物理学专业(射电天文方向)直博研究生,有扎实的物理、数学与统计学基础,
%擅长数据建模与分析,热衷计算机和网络技术,
%有 10 年的 Linux 和 BSD 使用经验,
%熟悉常用的命令行工具,熟练掌握 Shell、Python 和 C 语言编程。
%积极实践自由开源精神,
%在 \link{https://github.com/liweitianux}{GitHub} 上分享多个项目,
%是 \link{https://www.dragonflybsd.org}{DragonFly BSD} 操作系统的开发者,
%并积极参与其他多个开源项目。
%真诚应聘贵公司的\textbf{\getposition}职位,
%期待加入贵公司,帮助实现公司目标,同时获得自身成长。
%\par}  % NOTE: \par is needed

%======================================================================
% Competences / Skills & Languages
\sectionTitle{技能和语言}{\faWrench}
%======================================================================
\begin{competences}
  \comptence{\icon{\faCoffee} Java}{%
    %\icon{\faLinux} Linux (10 年);
    %\icon{\faFreebsd} BSD (DragonFly BSD 和 FreeBSD, 7 年)
    熟悉使用Maven构建项目;
    在Maven中心仓库中贡献一个\link{https://mvnrepository.com/artifact/com.github.zhangzhongjun/BilinearMapAccumulator}{开源库}
  }
  \comptence{\icon{\faLock}密码学}{%
    密码学基础知识;云存储,可搜索加密;以太坊;实现相关论文方案的能力
  }
  \comptence{\icon{\faCodeBranch} 版本控制}{
    熟悉Git技术;在github上分享多个仓库;收获一些星星和follower
  }
  \comptence{\icon{{\FA \symbol{"F717}}} 网络爬虫}{%
    熟悉各种反爬机制;
    20余\link{https://github.com/zhangzhongjun?utf8=crawler&tab=repositories&q=crawler&type=&language=}{爬虫项目};
    经营一家爬虫网店,完成50余订单,受到客户好评;
    熟悉Scrapy框架
  }
  \comptence{\icon{\faPython}脚本语言}{%
    Python, VBScript, Linux脚本
  }
  \comptence{\icon{\faBars}其他技能}{%
    CET-6,无障碍阅读DOC;证券从业资格证书持有人;PPT;PS
  }
\end{competences}
%
%%======================================================================
%% Education
%\sectionTitle{教育背景}{\faGraduationCap}
%%======================================================================
%\begin{educations}
%  \education
%    {2017.09}
%    [2020.06]
%    {西安电子科技大学}
%    {网络与信息安全学院}
%    {网络空间安全}%
%    %{硕士(在读,导师 陈晓峰 教授,研究方向为密码学、云计算安全)}
%    {硕士(在读,导师 陈晓峰 教授,密码学、云计算安全)}%
%
%  %\separator{0.2em}
%  \education
%    {2013.09}
%    [2017.06]
%    {山西大学}
%    {计算机科学与信息技术学院}
%    {计算机科学与技术}%
%    {学士(导师 曹峰 教授,研究方向为数据挖掘)}
%\end{educations}


%======================================================================
% Education
\sectionTitle{教育背景}{\faGraduationCap}
%======================================================================
\begin{myeducations}
  \myeducation
    {2017.09}
    [2020.06]
    {西安电子科技大学}
    %[网络与信息安全学院]
    {网络空间安全}%
    %{硕士(在读,导师 陈晓峰 教授,研究方向为密码学、云计算安全)}
    {硕士(在读,导师 陈晓峰 教授,密码学、云计算安全)}%

  \separator{0.2em}
  \myeducation
    {2013.09}
    [2017.06]
    {山西大学}
    [计算机科学与信息技术学院]
    {计算机科学与技术}%
    {学士(导师 曹峰 教授,研究方向为数据挖掘)}%
\end{myeducations}

%%======================================================================
%% Research Achievements
%\sectionTitle{科研成果}{\FAS \symbol{"F5D2}}
%%======================================================================
%\begin{itemize}
%  \item 参与研究课题:
%    \enquote{低频射电天空的高精度仿真与微弱天体辐射信号的识别}(重点项目)、
%    \enquote{星系和星系团的 X 射线研究、宇宙低频射电辐射研究}(杰出青年基金)
%  \item 开发低频射电天空图像模拟软件:
%    \link{https://github.com/liweitianux/fg21sim}{\texttt{FG21sim}}
%    (Python)
%  \item 开发程序帮助半自动化分析 \textit{Chandra} X~射线卫星观测数据:
%    \link{https://github.com/liweitianux/chandra-acis-analysis}{\texttt{chandra-acis-analysis}}
%    (Python, Shell, Tcl)
%  \item 利用卷积去噪自动编码器(CDAE)在观测频率维度有效分离微弱的
%    宇宙再电离(EoR)信号
%  \item 利用卷积神经网络(CNN)对 FIRST 巡天的射电星系图像
%    根据形态特征进行分类
%  \item 显著改进星系团射电晕的建模,并考虑低频干涉阵列的复杂仪器效应
%  \item 分析 200 多个星系团的 \textit{Chandra} 观测数据,
%    改进光谱拟合中各背景成分的建模,获到更准确可靠的拟合结果
%  \item 发表第一作者 SCI 论文 2 篇,合作 SCI 论文 8 篇
%\end{itemize}

%%======================================================================
%% Computer Skills
%\sectionTitle{计算机技能}{\faCode}
%%======================================================================
%\begin{itemize}
%  \item DragonFly BSD 操作系统开发者:
%    关注内核网络模块及工具,修正问题并改进
%  \item 使用 Ansible 配置和管理 VPS,部署个人域名邮箱、权威 DNS、
%    网站、Git、IRC 等服务
%  \item 搭建并管理课题组的工作站、计算集群(4 节点)和网络设备
%  \item 参与配置和测试上海天文台的 SKA 高性能计算集群原型机
%    (1 管理节点 + 1 存储节点 + 4 计算节点)
%  \item 为\enquote{2014 第一届中国--新西兰联合 SKA 暑期学校}
%    设计并开发网站(Django, Bootstrap, jQuery)
%\end{itemize}
%
%\clearpage


%======================================================================
% Internships
\sectionTitle{项目经历}{\faBriefcase}
%======================================================================
\begin{experiences}
  \experience
    {2018.12}%
    [2017.04]%
    %{面向隐私保护的密文搜索技术研究}%
    {前向安全的可搜索加密技术研究}%
    {算法设计, 实现及其工程应用}%
    [国家自然科学基金]%
    [\begin{itemize}
      \item{\icon{\faFlag}} 云存储是云计算的基础和核心技术,云存储的痛点之一就是如何兼顾安全性和可搜索性
      \item{\icon{\faFlag}} 密文搜索技术指客户在上传数据之前,首先对数据进行加密,且不影响搜索功能
      \item{\icon{\faCheck}} 完成一篇论文(第一作者);以第一作者申请一项专利;
      \item{\icon{\faCheck}} 基于lucene框架,使用Java,python,整理两个基准数据集 \githublinks{zhangzhongjun}{ProcessEnwiki} \githublinks{zhangzhongjun}{ProcessEDRM1}%:ENRON和enWiki
    \end{itemize}]%

  \separator{0.2em}
  \experience
    {2018.05}%
    [2017.09]%
    {信睿保护用户隐私的云存储系统}%
    {软件设计及实现}%
    [西安电子科技大学校企合作项目]%
    [\begin{itemize}
      \item{\icon{\faFlag}} 一个B/S架构的云存储系统,使用MySql作为后台数据库,使用Redis缓存数据
      \item{\icon{\faFlag}} 后端使用Django框架;前端使用Jquery+Bootstrap。
      \item{\icon{\faCheck}} 获得软著《信睿保护用户隐私的云存储系统》(第一作者)
      %\item{\icon{\faCheck}} 荣获2017年“全国密码学竞赛”三等奖
    \end{itemize}]%

  \separator{0.2em}
  \experience
    {}
    [2018.09]%
    {多关键词可搜索加密技术研究}%
    {算法设计, 实现}%
    [国家自然科学基金]%
    [\begin{itemize}
      \item{\icon{\faFlag}} 或搜索(包含关键词A或B)、与搜索(包含关键词A和B)、非搜索(包含关键词A而不包含B)
      \item{\icon{\faCheck}} 使用JAVA实现了密码学原语——基于双线性对的聚合器 \githublinks{zhangzhongjun}{BilinearMapAccumulator}
      \item{\icon{\faCheck}} 使用JAVA实现3个该领域研究最前沿方案   \githublinks{zhangzhongjun}{CashScheme} \githublinks{zhangzhongjun}{SunScheme} \githublinks{zhangzhongjun}{WangScheme}
    \end{itemize}]%


    \separator{0.2em}
    \experience
    {2019.03}
    [2018.06]%
    {以太坊底层研究与上层应用}%
    {阅读Ethereum黄皮书,实现Dapp}%
    [中国博士后科学基金特别资助]%
    [\begin{itemize}
      \item{\icon{\faFlag}} 着眼于密码学在以太坊中的使用,如账户地址涉及到的公钥计算,交易编排涉及到的Merkle树;同时把握其整体设计并追踪其最新发展动态
      \item{\icon{\faFlag}} 底层编码技术RLP,区块设计,叔块,账户状态,共识算法Ethash,EVM和字节码;
      \item{\icon{\faCheck}} 刨析了The DAO项目,了解了其经济学原理,对智能合约进行Code Review,分析其被攻击的原因
      %\item{\icon{\faCheck}} 编写智能合约,结合前端知识实现一个Dapp;
    \end{itemize}]%

\end{experiences}





%
%%======================================================================
%% Papers / Publications
%\sectionTitle{发表论文}{\faPhone}
%%======================================================================
%\begin{itemize}
%  \small
%  \item \textbf{Li,~W.}, Xu,~H., Ma,~Z., Zhu, R., Hu,~D., Zhu,~Z.,
%    Shan,~C., Zhu, J. \& Wu, X.-P.,
%    \enquote{\it Separating the EoR Signal with a Convolutional Denoising
%      Autoencoder: a Deep-learning-based Method,}
%    2018, Monthly Notices of the Royal Astronomical Society Letters
%    (under review; SCI; IF=4.96)
%  \item \textbf{Li,~W.}, Xu,~H., Ma,~Z., Hu,~D., Zhu,~Z., Shan,~C.,
%    Wang,~J., Gu,~J., Lian,~X., Zheng,~Q., Zhu, J. \& Wu, X.-P.,
%    \enquote{\it Contribution of Radio Halos to the Foreground for
%      SKA EoR Experiments,}
%    2018, The Astrophysical Journal (under review; SCI; IF=5.53)
%  \item Ma,~Z., Xu,~H., Zhu,~J., Hu,~D., \textbf{Li,~W.}, Shan,~C., Zhu,~Z.,
%    Lian,~X., Gu,~L., Liu,~C. \& Wu,~X.-P.,
%    \enquote{\it A Machine Learning Based Morphological Classification
%      of 14,251 Radio AGNs Selected from the Best--Heckman Sample,}
%    2018, The Astrophysical Journal Supplement Series
%    (in revision; SCI; IF=8.96)
%  \item Hu,~D., Xu,~H., Kang,~X., \textbf{Li,~W.}, Zhu,~Z., Ma,~Z.,
%    Shan,~C., Zhang,~Z., Gu,~L., Liu,~C. \& Wu,~X.-P.,
%    \enquote{\it A Study of the Merger History of the Galaxy Group
%      HCG~62 Based on X-ray Observations and SPH Simulations,}
%    2017, The Astrophysical Journal
%    (in revision; SCI; IF=5.53)
%  \item Zheng,~Q., Johnston-Hollitt,~M., Duchesne,~S. \& \textbf{Li,~W.},
%    \enquote{\it Detection of a Double Relic in the Torpedo Cluster:
%      SPT-Cl J0245-5302,}
%    2018, Monthly Notices of the Royal Astronomical Society, 479, 730
%    (SCI; IF=4.96)
%  \item Ma,~Z., Zhu,~J., \textbf{Li,~W.} \& Xu,~H.,
%    \enquote{\it An Approach to Detect Cavities in X-ray Astronomical
%      Images Using Granular Convolutional Neural Networks,}
%    2017, IEICE Transactions on Information and System, 100(10), 2578
%    (SCI; IF=0.41)
%  \item Zhang,~C., Xu,~H., Zhu,~Z., \textbf{Li,~W.}, Hu,~D., Wang,~J.,
%    Gu,~J., Gu,~L., Zhang,~Z., Liu,~C., Zhu,~J. \& Wu,~X.-P.,
%    \enquote{\it A Chandra Study of the Image Power Spectra of 41
%      Cool Core and Non-cool Core Galaxy Clusters,}
%    2016, The Astrophysical Journal, 823, 116 (SCI; IF=5.53)
%  \item (另有 3 篇合作 SCI 论文)
%\end{itemize}

%======================================================================
% Awards / Scholarships / Certificates
\sectionTitle{获奖及证书}{\faTrophy}
%======================================================================
\begin{entries}
  \entry{2018.10}%
    %{第十三届全国研究生数学建模竞赛 \textbullet 成功参与奖}
    {阿里巴巴天池polardb数据库性能大赛 | 89名/1653 | 单人参赛}%
    [\begin{itemize}
      \item{\icon{\faFlag}} 以 Optane SSD 为背景,实现高效的 KV 存储引擎
      \item{\icon{\faFlag}} JVM调优,native方法调用,基于4K对齐的方案设计与架构优化,直接内存
    \end{itemize}]%

  \entry{2018.05}%
    {华为codecraft算法比赛 | 44名 | 队长}%
    [\begin{itemize}
      \item{\icon{\faFlag}} 时间序列预测+背包问题\quad \githublink{zhangzhongjun}{CodeCraft}
    \end{itemize}]%

  \entry{2017.10}%
    {全国密码学竞赛 | 国家三等奖 | 队长 \quad \githublink{zhangzhongjun}{XRCloud}}%
  \entry{2014-2016}%
    {“CCF第7次CSP能力认证”全国排名前3 | “软件设计师”证书 | “系统集成管理工程师”证书}
  \entry{2013-2019}%
    {一次西安电子科技大学一等奖学金(一万)、多次山西大学一等奖学金、一次励志奖学金(一万)}
    \entry{2018.12}%
    {“证券从业资格”证书}
\end{entries}

\end{document}

%% EOF
