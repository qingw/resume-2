\documentclass[zh]{resume}

% File information shown at the footer of the last page
\fileinfo{%
  \faCopyright{} 2019 ZhongJun Zhang,
  \creativecommons{by}{4.0},
  \githublink{zhangzhongjun}{resume}
}

\name{中俊}{张}

%设置期望岗位和期望工作城市
%\setposition{云计算开发工程师}
%\setlocation{深圳}
%\tagline{\getposition{} @ \getlocation}
%\taglineicon{\faBinoculars}


\photo{7em}{photo}

\socialinfo{
  \mobile{188-3510-9707}
  \email{zhongjunz@aliyun.com}
  \github{zhangzhongjun} \\
  \university{西安电子科技大学}
  \degree{网络空间安全 \textbullet 硕士}
  \weixin{meitong1214} \\
  %\address{上海}
  %\home{湖南 \textbullet 邵阳}
  %\birthday{1991-09-26}
}

\begin{document}
\makeheader

%======================================================================
% Competences / Skills & Languages
\sectionTitle{技能和语言}{\faWrench}
%======================================================================
\begin{competences}
  \comptence{\icon{\faAngleRight } 密码学}{%
    密码学基础知识;~~云存储,可搜索加密;~~以太坊,区块链基础;~~实现相关论文方案的能力
  }

  \comptence{\icon{\faAngleRight } Java}{%
    了解JVM;~~
    了解SSH框架;~~
    熟悉使用Maven构建项目;~~
    在Maven中心仓库中贡献一个\link{https://mvnrepository.com/artifact/com.github.zhangzhongjun/BilinearMapAccumulator}{开源库}
  }

%  \comptence{\icon{\faCodeBranch} 版本控制}{
%    熟悉Git技术;~~在github上有多个repos;~~收获一些星星和follower
%  }

  \comptence{\icon{ \faAngleRight}生产工具}{%
     Rust, Python, Git, Jetbrains系列IDE, 爬虫, VBA脚本, linux脚本
  }
    %\icon{{\FA \symbol{"F717}}}
%  \comptence{\icon{{\FA \symbol{"F717}}}网络爬虫}{%
 %   熟悉各种反爬机制;~~
%    20余\link{https://github.com/zhangzhongjun?utf8=crawler&tab=repositories&q=crawler&type=&language=}{爬虫项目};~~
 %   经营\link{https://m.zbj.com/shop/13255669/evaluation.html}{爬虫网店},完成50余订单,受到客户好评;~~
 %   熟悉Scrapy框架
 % }

%  \comptence{\icon{\faBars}其他技能}{%
%    CET-6;~~证券从业资格证书持有人;~~PPT;~~PS
%  }
\end{competences}

%======================================================================
% Education
\sectionTitle{教育背景}{\faGraduationCap}
%======================================================================
\begin{myeducations}
  \myeducation
    {2017.09}
    [2020.06]
    {西安电子科技大学}
    %[网络与信息安全学院]
    {网络空间安全}%
    %{硕士(在读,导师 陈晓峰 教授,研究方向为密码学、云计算安全)}
    {硕士 \textbullet 陈晓峰 教授 \textbullet 密码学、云计算安全 \textbullet 排名9/69}%

  \separator{0.1em}
  \myeducation
    {2013.09}
    [2017.06]
    {山西大学}
    [计算机科学与信息技术学院]
    {计算机科学与技术}%
    {学士 \textbullet 曹峰 \textbullet 数据挖掘}%
\end{myeducations}


%======================================================================
% Internships
\sectionTitle{实习经历}{\faBriefcase}
%======================================================================
\begin{experiences}
  \experience
    {2019.09}%
    [2019.07]%
    {深圳前海微众银行股份有限公司}%
    {分布式商业科技发展部}%
    [区块链技术岗(深圳)]%
    [\begin{itemize}
      \item{\icon{\faFlag}} 调研了hyperledger在选择披露中的工作,code review了hyperledger-indy项目
      \item{\icon{\faFlag}} 实现了DPKI体系中的证书撤销功能,并开发了相关组件 \githublinks{HaoXuan40404}{indy-crypto/tree/accumulator}
    \end{itemize}]%

\end{experiences}


%======================================================================
% Education
\sectionTitle{科研成果}{\faAtom}
%======================================================================
\begin{myresearchs}
  \myresearch
    {\textbf{Towards Efficient Verifiable Forward Secure Searchable Symmetric Encryption}}
    {\textbf{ESORICS 2019 (CCF/CACR B类会议)}}
    {\textbf{Accepted}}
    {\textbf{第一作者}}%

  \myresearch
    {Efficient Verifiable Multi-Key Searchable Encryption in Cloud Computing}
    {ACCESS}
    {Under Review}
    { }

  \myzhuanli
    {《一种支持验证的前向安全可搜索加密存储系统及方法》}
    {学生第一作者}

  \myzhuanli
    {《一种多用户场景下可验证的多密钥可搜索加密方法》}
    {合作作者}

  \myruanzhu
    {《信睿保护用户隐私的云存储系统》}
    {第一作者}

\end{myresearchs}





%======================================================================
% Internships
\sectionTitle{项目经历}{\faCode}
%======================================================================
\begin{experiences}
  \experience
    {2018.12}%
    [2017.04]%
    {云环境中外包数据可验证存储关键技术研究}%
    {算法设计, 实现及应用}%
    [国家自然科学基金项目]%
    [\begin{itemize}
      \item{\icon{\faFlag}} 云存储是云计算的基础和核心技术,云存储的痛点之一就是如何兼顾安全性和可搜索性
      \item{\icon{\faFlag}} 可搜索加密是指一种新型加密技术,可以在使用该加密手段加密的密文上直接搜索
      \item{\icon{\faCheck}} 基于Lucene,使用Java,Python,整理两个基准数据集 \githublinks{zhangzhongjun}{ProcessEnwiki} \githublinks{zhangzhongjun}{ProcessEDRM1}%:ENRON和enWiki
    \end{itemize}]%

  \separator{0.1em}
%
%  \experience
%    {}
%    [2018.09]%
%    {多关键词可搜索加密技术研究}%
%    {算法设计, 实现}%
%    [国家自然科学基金]%
%    [\begin{itemize}
%      \item{\icon{\faFlag}} 或搜索(包含关键词A或B)、与搜索(包含关键词A和B)、非搜索(包含关键词A而不包含B)
%      \item{\icon{\faCheck}} 使用Java实现了密码学原语——基于双线性对的聚合器 \githublinks{zhangzhongjun}{BilinearMapAccumulator}
%      \item{\icon{\faCheck}} 使用Java实现3个该领域研究最前沿方案   \githublinks{zhangzhongjun}{CashScheme} \githublinks{zhangzhongjun}{SunScheme} \githublinks{zhangzhongjun}{WangScheme}
%    \end{itemize}]%


  \separator{0.1em}
  \experience
    {2018.05}%
    [2017.09]%
    {信睿保护用户隐私的云存储系统}%
    {软件设计及实现}%
    [西安电子科技大学校企合作项目]%
    [\begin{itemize}
      \item{\icon{\faFlag}} 一个B/S架构的云存储系统,使用MySql作为后台数据库,使用Redis缓存数据
      \item{\icon{\faFlag}} 后端使用Django框架;前端使用Jquery+Bootstrap。
      \item{\icon{\faCheck}} 项目地址(可点击查看): \githublink{zhangzhongjun}{XRCloud}
      %\item{\icon{\faCheck}} 荣获2017年“全国密码学竞赛”三等奖
    \end{itemize}]%



 %   \separator{0.2em}
%    \experience
%    {2019.03}
%    [2018.06]%
%    {以太坊底层研究与上层应用}%
%    {阅读Ethereum黄皮书,实现Dapp}%
%    [中国博士后科学基金特别资助]%
%    [\begin{itemize}
%      \item{\icon{\faFlag}} 着眼于密码学在以太坊中的使用,如账户地址涉及到的公钥计算,交易编排涉及到的Merkle树;同时把握其整体设计并追踪其最新发展动态
%      \item{\icon{\faFlag}} 底层编码技术RLP,区块设计,叔块,账户状态,共识算法Ethash,EVM和字节码;
%      \item{\icon{\faCheck}} 刨析了The DAO项目,了解了其经济学原理,对智能合约进行Code Review,分析其被攻击的原因
%      %\item{\icon{\faCheck}} 编写智能合约,结合前端知识实现一个Dapp;
%    \end{itemize}]%

\end{experiences}


%
%%======================================================================
%% Papers / Publications
%\sectionTitle{发表论文}{\faPhone}
%%======================================================================
%\begin{itemize}
%  \small
%  \item \textbf{Li,~W.}, Xu,~H., Ma,~Z., Zhu, R., Hu,~D., Zhu,~Z.,
%    Shan,~C., Zhu, J. \& Wu, X.-P.,
%    \enquote{\it Separating the EoR Signal with a Convolutional Denoising
%      Autoencoder: a Deep-learning-based Method,}
%    2018, Monthly Notices of the Royal Astronomical Society Letters
%    (under review; SCI; IF=4.96)
%  \item \textbf{Li,~W.}, Xu,~H., Ma,~Z., Hu,~D., Zhu,~Z., Shan,~C.,
%    Wang,~J., Gu,~J., Lian,~X., Zheng,~Q., Zhu, J. \& Wu, X.-P.,
%    \enquote{\it Contribution of Radio Halos to the Foreground for
%      SKA EoR Experiments,}
%    2018, The Astrophysical Journal (under review; SCI; IF=5.53)
%  \item Ma,~Z., Xu,~H., Zhu,~J., Hu,~D., \textbf{Li,~W.}, Shan,~C., Zhu,~Z.,
%    Lian,~X., Gu,~L., Liu,~C. \& Wu,~X.-P.,
%    \enquote{\it A Machine Learning Based Morphological Classification
%      of 14,251 Radio AGNs Selected from the Best--Heckman Sample,}
%    2018, The Astrophysical Journal Supplement Series
%    (in revision; SCI; IF=8.96)
%  \item Hu,~D., Xu,~H., Kang,~X., \textbf{Li,~W.}, Zhu,~Z., Ma,~Z.,
%    Shan,~C., Zhang,~Z., Gu,~L., Liu,~C. \& Wu,~X.-P.,
%    \enquote{\it A Study of the Merger History of the Galaxy Group
%      HCG~62 Based on X-ray Observations and SPH Simulations,}
%    2017, The Astrophysical Journal
%    (in revision; SCI; IF=5.53)
%  \item Zheng,~Q., Johnston-Hollitt,~M., Duchesne,~S. \& \textbf{Li,~W.},
%    \enquote{\it Detection of a Double Relic in the Torpedo Cluster:
%      SPT-Cl J0245-5302,}
%    2018, Monthly Notices of the Royal Astronomical Society, 479, 730
%    (SCI; IF=4.96)
%  \item Ma,~Z., Zhu,~J., \textbf{Li,~W.} \& Xu,~H.,
%    \enquote{\it An Approach to Detect Cavities in X-ray Astronomical
%      Images Using Granular Convolutional Neural Networks,}
%    2017, IEICE Transactions on Information and System, 100(10), 2578
%    (SCI; IF=0.41)
%  \item Zhang,~C., Xu,~H., Zhu,~Z., \textbf{Li,~W.}, Hu,~D., Wang,~J.,
%    Gu,~J., Gu,~L., Zhang,~Z., Liu,~C., Zhu,~J. \& Wu,~X.-P.,
%    \enquote{\it A Chandra Study of the Image Power Spectra of 41
%      Cool Core and Non-cool Core Galaxy Clusters,}
%    2016, The Astrophysical Journal, 823, 116 (SCI; IF=5.53)
%  \item (另有 3 篇合作 SCI 论文)
%\end{itemize}

%======================================================================
% Awards / Scholarships / Certificates
\sectionTitle{获奖及证书}{\faTrophy}
%======================================================================
\begin{entries}

  \entry{2018.05}%
    {2018年华为软件精英挑战赛(Code Craft 2018)  \quad | \quad 西北赛区44名  \quad | \quad 队长}%
  %  [\begin{itemize}
 %     \item{\icon{\faFlag}} 时间序列预测+背包问题\quad \githublink{zhangzhongjun}{CodeCraft}
  %  \end{itemize}]%

  \entry{2018.10}%
    %{第十三届全国研究生数学建模竞赛 \textbullet 成功参与奖}
    {第一届阿里巴巴天池POLARDB数据库性能大赛  \quad | \quad 全国百强(共1653支队伍参赛)  \quad | \quad 单人参赛}%
   % [\begin{itemize}
 %     \item{\icon{\faFlag}} 以 Optane SSD 为背景,实现高效的 KV 存储引擎
   %   \item{\icon{\faFlag}} JVM调优,native方法调用,基于4K对齐的方案设计与架构优化,直接内存
   % \end{itemize}]%

  \entry{2015.10}%
    %{第十三届全国研究生数学建模竞赛 \textbullet 成功参与奖}
    {2015年华北五省(市、自治区)及港澳台大学生计算机应用大赛  \quad | \quad 国家一等奖 }%
	\entry{2019.7}%
    {第四届全国高校密码数学挑战赛  \quad | \quad 国家三等奖}%
  \entry{2017.10}%
    {第三届全国密码技术竞赛  \quad | \quad 国家三等奖  \quad | \quad 队长  \quad \githublink{zhangzhongjun}{XRCloud}}%
  \entry{2014-2016}%
    {“CCF第7次CSP能力认证”全国排名前3\% | “软件设计师”证书 | “系统集成管理工程师”证书}
  \entry{2013-2019}%
    {西安电子科技大学校级学业一等奖学金(一万)、多次山西大学校级学业一等奖学金、励志奖学金(一万)、优秀研究生称号}
    \entry{2018.12}%
    {“证券从业资格”证书}
\end{entries}

\end{document}

%% EOF
